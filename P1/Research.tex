%! TeX root = Research.tex    % <-- this tells VimTeX this is the main file

\documentclass[11pt,a4paper]{article}

\usepackage[utf8]{inputenc}
\usepackage[T1]{fontenc}
\usepackage{amsmath}
\usepackage{amssymb}
\usepackage{geometry}
\geometry{margin=2.5cm}

\title{COS333 Practical 1}
\author{Dewald Colesky u23536030}
\date{2 March 2026}

\begin{document}

\maketitle
\section{Research Questions}
\section{Implementation Questions}


\newpage
\section{Research Questions}

\begin{enumerate}
    \item A computer programming language designed to experiment with weird ideas. They may be difficult to program in and are often created with a specific goal in mind (As a joke, to explore concepts, to be 'artistic', etc.)

  \item Useful contributions of esoteric programming languages:
\begin{itemize}
      \item Test the boundaries and possibilities of a programming concept
      \item Acts as a challenge for programmers
      \item Can teach programming concepts minimally (like Turing completeness)
      \item Exploring language design and minimalism
\end{itemize}
    \item CHOSEN ESOTERIC LANGUAGES (WHITESPACE, INTERCAL)

\end{enumerate}

\newpage
Hello world in Whitespace (The following source code is for a Whitespace "Hello, world!" program. For clarity, it is annotated with S, T and L before each space, tab, and linefeed.)
\begin{verbatim}
S S S T	S S T	S S S L:Push_+1001000=72='H'_onto_the_stack
T	L
S S :Output_'H';_S S S T	T	S S T	S T	L:Push_+1100101=101='e'_onto_the_stack
T	L
S S :Output_'e';_S S S T	T	S T	T	S S L:+1101100=108='l'
T	L
S S S S S T	T	S T	T	S S L:+1101100=108='l'
T	L
S S S S S T	T	S T	T	T	T	L:+1101111=111='o'
T	L
S S S S S T	S T	T	S S L:+101100=44=','
T	L
S S S S S T	S S S S S L:+100000=32=Space
T	L
S S S S S T	T	T	S T	T	T	L:+1110111=119='w'
T	L
S S S S S T	T	S T	T	T	T	L:+1101111=111='o'
T	L
S S S S S T	T	T	S S T	S L:+1110010=114='r'
T	L
S S S S S T	T	S T	T	S S L:+1101100=108='l'
T	L
S S S S S T	T	S S T	S S L=+1100100=100='d'
T	L
S S S S S T	S S S S T	L:+100001=33='!'
T	L
S S :Output_'!';_L
L
L:End_the_program
\end{verbatim}

The following is a "Hello, world!" program in INTERCAL. 
\begin{verbatim}
DO ,1 <- #13
PLEASE DO ,1 SUB #1 <- #238
DO ,1 SUB #2 <- #108
DO ,1 SUB #3 <- #112
DO ,1 SUB #4 <- #0
DO ,1 SUB #5 <- #64
DO ,1 SUB #6 <- #194
DO ,1 SUB #7 <- #48
PLEASE DO ,1 SUB #8 <- #22
DO ,1 SUB #9 <- #248
DO ,1 SUB #10 <- #168
DO ,1 SUB #11 <- #24
DO ,1 SUB #12 <- #16
DO ,1 SUB #13 <- #162
PLEASE READ OUT ,1
PLEASE GIVE UP
\end{verbatim}


\section{Implementation Questions}

    \begin{thebibliography}{9}

        \bibitem{singer2025}
        Jeremy Singer and Steve Draper.
        \newblock Let's Take Esoteric Programming Languages Seriously.
        \newblock In \emph{Proceedings of Onward! '25}, 2025.
        \newblock \url{https://dl.acm.org/doi/10.1145/3759429.3762632}

        \bibitem{temkin2025book}
        Daniel Temkin.
        \newblock \emph{Forty-Four Esolangs: The Art of Esoteric Code}.
        \newblock MIT Press, 2025.
        \newblock \url{https://mitpress.mit.edu/9780262553087/forty-four-esolangs/}

        \bibitem{temkinieee2025}
        Daniel Temkin.
        \newblock Esoteric Programming Languages: A Unique Challenge.
        \newblock IEEE Spectrum, September 2025.
        \newblock \url{https://spectrum.ieee.org/esoteric-programming-languages-daniel-temkin}

https://en.wikipedia.org/wiki/Whitespace_(programming_language)
https://en.wikipedia.org/wiki/INTERCAL


    \end{thebibliography}



\end{document}
